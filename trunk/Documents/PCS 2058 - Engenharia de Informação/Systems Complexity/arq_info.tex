A (r)evolu��o da Computa��o, ao contr�rio das ``ci�ncias exatas", n�o ocorreu devido a \emph{busca do conhecimento pelo conhecimento}, mas devido a necessidade de transmitir e processar informa��es de maneira r�pida, segura e precisa: os requisitos dos c�lculos bal�sticos do primeiro computador, o ENIAC, s�o exemplos dessas caracter�sticas.

Os aspectos de \emph{velocidade} e \emph{precis�o}, particularmente, est�o fortemente associados a estruturas f�sicas, de \emph{hardware}, fracamente associados �s informa��es (ou dados) do sistema.

Contudo, a \emph{seguran�a (dos dados)} � uma caracter�stica fortemente associada � informa��o transmitida (embora ainda conectada ao \emph{hardware}), seja pelo seu \emph{tamanho}, seja pelo seu \emph{valor agregado}. Em particular, esta �ltima caracter�stica vem se tornando o principal agente de 
